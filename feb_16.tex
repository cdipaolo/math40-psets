\documentclass[12pt,letterpaper]{hmcpset}
\usepackage[margin=1in]{geometry}
\usepackage{graphicx}
\usepackage{amsmath}

% info for header block in upper right hand corner
\name{ }
\class{Math 40 - Section ---}
\assignment{HW 7 - Linear Transformations and Determinants}
\duedate{Tuesday, February 9, 2016}

\newcommand{\pn}[1]{\left( #1 \right)}
\newcommand{\abs}[1]{\left| #1 \right|}
\newcommand{\bk}[1]{\left[ #1 \right]}

\newcommand{\vb}{\mathbf{v}}
\newcommand{\ub}{\mathbf{u}}
\renewcommand{\labelenumi}{{(\alph{enumi})}}

\newcommand{\m}[1]{\begin{bmatrix} #1 \end{bmatrix}}

\begin{document}

\problemlist{3.6.\{6, 10, 20, 54\}, 4.2.\{8, 33\}}

% 3.6.6 %
\begin{problem}[3.6.6]
    Prove that the given transformation
    is a linear transformation, using
    the definition (or the Remark following
    Example 3.55):

    \[
        T\m{x\\y\\z} = \m{x+z\\y+z\\x+y}.
    \]
\end{problem}

\begin{solution}
\vfill
\end{solution}
\newpage

% 3.6.10 %
\begin{problem}[3.6.10]
    Give a counterexample to show that
    the given transformation is not a
    linear transformation:

    \[
        T\m{x\\y} = \m{x+1\\y-1}.
    \]
\end{problem}

\begin{solution}
\vfill
\end{solution}
\newpage

% 3.6.20 %
\begin{problem}[3.6.20]
    Find the standard matrix of the
    linear transformation from
    $\mathbb{R}^2$ to $\mathbb{R}^2$
    which performs a counterclockwise
    rotation through $120^o$ about
    the origin.
\end{problem}

\begin{solution}
\vfill
\end{solution}
\newpage

% 3.6.54 %
\begin{problem}[3.6.54]
    Prove that (as noted at the 
    beginning of this section)
    the range of a linear transformation 
    $T : \mathbb{R}^n \mapsto \mathbb{R}^m$
    is the column space of its matrix $[T]$.
\end{problem}

\begin{solution}
\vfill
\end{solution}
\newpage

% 4.2.8 %
\begin{problem}[4.2.8]
    Compute the determinant of the following
    matrix using cofactor expansion along
    any row or column that seems convenient.
    \[
        \left| \begin{matrix}
            1&1&-1\\
            2&0&1\\
            3&-2&1
        \end{matrix} \right|
    \]
\end{problem}

\begin{solution}
\vfill
\end{solution}
\newpage

% 4.2.33 %
\begin{problem}[4.2.33]
    Use properties of determinants to
    evaluate the determinant by inspection.
    Explain your reasoning.
    \[
        \left| \begin{matrix}
            0&2&0&0\\
            -3&0&0&0\\
            0&0&0&4\\
            0&0&1&0
        \end{matrix} \right|
    \]
\end{problem}

\begin{solution}
\vfill
\end{solution}

\end{document}
