\documentclass[12pt,letterpaper]{hmcpset}
\usepackage[margin=1in]{geometry}
\usepackage{graphicx}
\usepackage{amsmath}

% info for header block in upper right hand corner
\name{ }
\class{Math 40 - Section ---}
\assignment{HW 9 - Eigenvalues and Eigenvectors}
\duedate{Tuesday, February 23, 2016}

\newcommand{\pn}[1]{\left( #1 \right)}
\newcommand{\abs}[1]{\left| #1 \right|}
\newcommand{\bk}[1]{\left[ #1 \right]}

\renewcommand{\labelenumi}{{(\alph{enumi})}}
\newcommand{\dm}[1]{\left| \renewcommand*{\arraystretch}{1.3} \begin{matrix} #1 \end{matrix} \right|}
\newcommand{\m}[1]{\renewcommand*{\arraystretch}{1.3} \begin{bmatrix} #1 \end{bmatrix}}
\newcommand{\am}[2]{
  \left[\renewcommand*{\arraystretch}{1.3}
    \begin{matrix}
      #1
    \end{matrix}
  \right| \left.
  \renewcommand*{\arraystretch}{1.3}
  \begin{matrix}
    #2
  \end{matrix} \right]} % augmented matrix

\begin{document}

\problemlist{4.3.\{6, 8, 22, 24, 32, 34\}}

% 4.3.6 %
\begin{problem}[4.3.6]
	Compute (a) the characteristic polynomial of A, (b) the eigenvalues of A, (c) a basis for each eigenspace of A, and (d) the algebraic and geometric multiplicity of each eigenvalue.
  $$ A = \m{1 & 0 & 2 \\ 3 & -1 & 3 \\ 2 & 0 & 1} $$
\end{problem}

\begin{solution}
\vfill
\end{solution}
\newpage

% 4.3.8 %
\begin{problem}[4.3.8]
	Compute (a) the characteristic polynomial of A, (b) the eigenvalues of A, (c) a basis for each eigenspace of A, and (d) the algebraic and geometric multiplicity of each eigenvalue.
  $$ A = \m{1 & -1 & -1 \\ 0 & 2 & 0 \\ -1 & -1 & 1} $$
\end{problem}

\begin{solution}
\vfill
\end{solution}
\newpage

% 4.3.22 %
\begin{problem}[4.3.22]
	If $\mathbf{v}$ is an eigenvectur of $A$ with corresponding eigenvalue $\lambda$ and $c$ is a scalar, show that $\mathbf{v}$ is an eigenvector of $A - cI$ with corresponding eigenvalue $\lambda - c$.
\end{problem}

\begin{solution}
\vfill
\end{solution}
\newpage

% 4.3.24 %
\begin{problem}[4.3.24]
Let $A$ and $B$ be $n \times n$ matrices with eigenvalues $\lambda$ and $\mu$, respectively.
	\begin{enumerate}
		\item
			Give an example to show that $\lambda + \mu$ need not be an eigenvalue of $A+B$.
		\item
			Give an example to show that $\lambda\mu$ need not be an eigenvalue of $AB$.
		\item
			Suppose $\lambda$ and $\mu$ correspond to the \emph{same} eigenvector $\mathbf{x}$. Show that, in this case, $\lambda + \mu$ is an eigenvalue of $A+B$ and $\lambda\mu$ is an eigenvalue of $AB$.
	\end{enumerate}

\end{problem}

\begin{solution}
\vfill
\end{solution}
\newpage

% 4.3.32 %
\begin{problem}[4.3.32]
	\begin{enumerate}
		\item
			Use mathematical induction to prove that, for $n \geq 2$, the companion matrix $C(p)$ of $p(c) = x^n + a_{n-1}x^{n-1} + \dots + a_1x + a_0$ has characteristic polynomial $(-1)^np(\lambda)$. [\emph{Hint}: Expand by cofactors along the last column. You may find it helpful to introduce the polynomial $q(x) = (p(x) -a_0)/x$.]
		
		\item
			Show that if $\lambda$ is an eigenvalue of the companion matrix $C(p)$ in Equation (4), then an eigenvector corresponding to $\lambda$ is given by
			$$\m{\lambda^{n-1}\\\lambda^{n-2}\\ \vdots \\ \lambda \\ 1}$$
	\end{enumerate}

\end{problem}

\begin{solution}
\vfill
\end{solution}
\newpage

% 4.3.34 %
\begin{problem}[4.3.34]
Verify the Cayley-Hamilton Theorem for
$$ A = \m{1 & 1 & 0 \\ 1 & 0 & 1 \\ 0 & 1 & 1}$$
\end{problem}

\begin{solution}
\vfill
\end{solution}

\end{document}
