\documentclass[12pt,letterpaper]{hmcpset}
\usepackage[margin=1in]{geometry}
\usepackage{graphicx}
\usepackage{amsmath}

% info for header block in upper right hand corner
\name{ --- }
\class{Math 40}
\assignment{Methods for Solving Linear Systems}
\duedate{Friday, January 29, 2016}

\newcommand{\pn}[1]{\left( #1 \right)}
\newcommand{\abs}[1]{\left| #1 \right|}
\newcommand{\bk}[1]{\left[ #1 \right]}

\newcommand{\vb}{\mathbf{v}}
\newcommand{\ub}{\mathbf{u}}
\newcommand{\fx}{f \left( x \right) =}
\newcommand*\LH{\ensuremath{\overset{\kern2pt L'H}{=}}}
\renewcommand{\labelenumi}{{(\alph{enumi})}}

\begin{document}

\problemlist{2.2.\{12, 14, 26, 28, 30, 42\}}

% 2.2.12 %
\begin{problem}[2.2.12]
Use elementary row operations to reduce the given matrix to (a) row echelon form and (b) reduced row echelon form. $$ \begin{bmatrix}
	2 & -4 & -2 & 6\\
	3 & 1 & 6 & 6
\end{bmatrix}$$

\end{problem}

\begin{solution}
\vfill
\end{solution}

\newpage

% 2.2.14 %
\begin{problem}[2.2.14]

Use elementary row operations to reduce the given matrix to (a) row echelon form and (b) reduced row echelon form. $$ \begin{bmatrix}
	-2 & -4 & 7\\
	-3 & -6 & 10\\
	1 & 2 & -3
\end{bmatrix}$$

\end{problem}

\begin{solution}
\vfill
\end{solution}
\newpage

% 2.2.26 %
\begin{problem}[2.2.26]
Solve the given system of equations using either Gaussian or Guass-Jordan elimination.
	\begin{align*}
		x - y + z &= 0\\
		-x + 3y + z &= 5\\
		3x + y + 7z &= 2
	\end{align*}
\end{problem}

\begin{solution}
\vfill
\end{solution}
\newpage

% 2.2.28 %
\begin{problem}[2.2.28]

Solve the given system of equations using either Gaussian or Guass-Jordan elimination.
	\begin{align*}
		2w + 3x - y + 4z &= 1\\
		3w - x + z &= 1\\
		3w - 4x + y - z &= 2
	\end{align*}

\end{problem}

\begin{solution}
\vfill
\end{solution}
\newpage

% 2.2.30 %
\begin{problem}[2.2.30]

Solve the given system of equations using either Gaussian or Guass-Jordan elimination.
	\begin{align*}
		-x_1 + 3x_2 - 2x_3 + 4x_4 &= 0\\
		2x_1 - 6x_2 + x_3 - 2x_4 &= -3\\
		x_1 - 3x_2 + 4x_3 - 8x_4 &= 2
	\end{align*}
	
\end{problem}

\begin{solution}
\vfill
\end{solution}
\newpage

% 2.2.42 %
\begin{problem}[2.2.42]

For what value(s) of $k$, if any, will the systems have (a) no solution, (b) a unique solution, and (c) infinitely many solutions?
	\begin{align*}
		x - 2y + 3z &= 2\\
		x + y + z &= k\\
		2x - y + 4z &= k^2
	\end{align*}
\end{problem}

\begin{solution}
\vfill
\end{solution}

\end{document}
