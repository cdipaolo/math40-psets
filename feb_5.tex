\documentclass[12pt,letterpaper]{hmcpset}
\usepackage[margin=1in]{geometry}
\usepackage{graphicx}
\usepackage{amsmath}

% info for header block in upper right hand corner
\name{ }
\class{Math 40}
\assignment{Matrix Algebra and the Inverse of a Matrix}
\duedate{Friday, February 5, 2016}

\newcommand{\pn}[1]{\left( #1 \right)}
\newcommand{\abs}[1]{\left| #1 \right|}
\newcommand{\bk}[1]{\left[ #1 \right]}

\newcommand{\vb}{\mathbf{v}}
\newcommand{\ub}{\mathbf{u}}
\newcommand{\fx}{f \left( x \right) =}
\newcommand*\LH{\ensuremath{\overset{\kern2pt L'H}{=}}}
\renewcommand{\labelenumi}{{(\alph{enumi})}}

\begin{document}

\problemlist{3.2.\{33, 36\}, 3.3.\{19, 42, 46, 53\}}

% 3.2.33 %
\begin{problem}[3.2.33]
Using induction, prove that for all $n \geq 1$, $$(A_1A_2 \cdot \cdot \cdot A_n)^T  = A_n^T \cdot \cdot \cdot A_2^TA_1^T$$
\end{problem}

\begin{solution}
\vfill
\end{solution}
\newpage

% 3.2.36 %
\begin{problem}[3.2.36]
\begin{enumerate}
	\item
		Give an example to show that if $A$ and $B$ are symmetric $n\times n$ matrices, then $AB$ need not be symmetric.
	\item
		Prove that if $A$ and $B$ are symmetric $n\times n$ matrices, then $AB$ is symmetric if and only if $AB = BA$.
\end{enumerate}

\end{problem}

\begin{solution}
\vfill
\end{solution}
\newpage

% 3.3.19 %
\begin{problem}[3.3.19]
Give a counterexample to show that $(A + B)^{-1} \neq A^{-1} + B^{-1}$ in general.

\end{problem}

\begin{solution}
\vfill
\end{solution}
\newpage

% 3.3.42 %
\begin{problem}[3.3.42]
\begin{enumerate}
	\item
		Prove that if $A$ is invertible and $AB = O$, then $B = O$.
	\item
		Give a counterexample to show that the result in part (a) may fail if $A$ is not invertible.
\end{enumerate}

\end{problem}

\begin{solution}
\vfill
\end{solution}

\newpage

% 3.3.46 %
\begin{problem}[3.3.46]
Prove that if a symmetric matrix is invertible, then its inverse is symmetric also.
\end{problem}

\begin{solution}
\vfill
\end{solution}

\newpage

% 3.3.53 %
\begin{problem}[3.3.53]
Use the Gauss-Jordan method to find the inverse of the given matrix (if it exists).
$$\begin{bmatrix}
	1 & -1 & 2 \\
	3 & 1 & 2 \\
	2 & 3 & -1
	\end{bmatrix}$$
\end{problem}

\begin{solution}
\vfill
\end{solution}
	
\end{document}
