\documentclass[12pt,letterpaper]{../hmcpset}
\usepackage[margin=1in]{geometry}
\usepackage{graphicx}
\usepackage{amsmath}

\newcommand{\pn}[1]{\left( #1 \right)}
\newcommand{\abs}[1]{\left| #1 \right|}
\newcommand{\bk}[1]{\left[ #1 \right]}
\newcommand{\vc}[1]{\left\langle #1 \right\rangle}
\newcommand{\pder}[2]{\frac{\partial #1}{\partial #2}}

% set numbering style for enumerated lists to be of form (a), (b), (c), etc.
\renewcommand{\labelenumi}{{(\alph{enumi})}}

% command to make 2in wide image centered on page
\newcommand{\diagram}[1]{\begin{center}\includegraphics[width=2in]{#1}\end{center}}

\begin{document}
	
	% info for header block in upper right hand corner
	\vspace*{-3\baselineskip}
	\begin{center}
		\begin{flushright} --- \\ Math 40 \\ FP1 - Vectors and Vector Spaces \\ Fri. 1/20, 2016 \\ 
		\end{flushright} 	
	\end{center}
	\bigskip

\problemlist{1.2\{ 17, 18, 52, 56, 58, 60 \}}

% 1.2.17
\begin{problem}[17]
	If \textbf{u}, \textbf{v}, and \textbf{w} are vectors in $\mathbb{R}^n$, \textit{n} $\geq$ 2, and c is a scalar, explain why the following expressions make no sense:
	
	\begin{enumerate}
		\item 
			$\| \mathbf{u} \cdot \mathbf{v} \|$
		\item
			$\mathbf{u} \cdot (\mathbf{v} \cdot \mathbf{w})$
		\item
			$\mathbf{u} \cdot \mathbf{v} + \mathbf{w}$
		\item
			$c \cdot (\mathbf{u} + \mathbf{w})$
	\end{enumerate}
\end{problem}

\begin{solution}
\end{solution}
\newpage

% 1.2.18
\begin{problem}[18]
	Determine whether the angle between $\mathbf{u} = \begin{bmatrix}
		3 \\
		0
	\end{bmatrix}$ and $\mathbf{v} = \begin{bmatrix}
		-1 \\
		1
	\end{bmatrix}$ is acute, obtuse or a right angle.
\end{problem}

\begin{solution}
\end{solution}
\newpage

% 1.2.52
\begin{problem}[52]
	Under what conditions are the following true for vectors \textbf{u} and \textbf{v} in $\mathbb{R}^2$ or $\mathbb{R}^3$?
	\begin{enumerate}
		\item 
			$\| \mathbf{u} + \mathbf{v} \| = \| \mathbf{u} \| + \| \mathbf{v} \|$
		\item
			$\| \mathbf{u} + \mathbf{v} \| = \| \mathbf{u} \| - \| \mathbf{v} \|$
	\end{enumerate}
\end{problem}

\begin{solution}
\end{solution}
\newpage

% 1.2.56
\begin{problem}[56]
	Prove $d(\mathbf{u}, \mathbf{w}) \leq d(\mathbf{u}, \mathbf{v}) + d(\mathbf{v}, \mathbf{w})$ for all vectors \textbf{u}, \textbf{v}, and \textbf{w}. 
\end{problem}

\begin{solution}
\end{solution}
\newpage

% 1.2.58
\begin{problem}[58]
	Prove  that $\mathbf{u} \cdot \mathit{c}\mathbf{v} = \mathit{c}(\mathbf{u} \cdot \mathbf{v})$ for all vectors \textbf{u} and \textbf{v} in $\mathbb{R}^n$ and all scalars \textit{c}.
\end{problem}

\begin{solution}
\end{solution}
\newpage
	
% 1.2.60
\begin{problem}[60]
	Suppose we know that $\mathbf{u} \cdot \mathbf{v} = \mathbf{u} \cdot \mathbf{w}$. Does it follow that $\mathbf{v} = \mathbf{w}$?
	
	If it does, give a proof that is valid in $\mathbb{R}^n$; otherwise, give a \textit{counterexample} (i.e., a \textit{specific} set of vectors \textbf{u}, \textbf{v}, and \textbf{w} for which $\mathbf{u} \cdot \mathbf{v} = \mathbf{u} \cdot \mathbf{w}$ but $\mathbf{v} \neq \mathbf{w}$). 
\end{problem}

\begin{solution}
\end{solution}
\newpage

\end{document}